% !TeX program = xelatex
\documentclass[12pt]{ctexart}
\usepackage[margin=1in]{geometry}
\usepackage{graphicx}
\usepackage{booktabs}
\usepackage{longtable}
\usepackage{array}
\usepackage{caption}
\usepackage{subcaption}
\usepackage{enumitem}
\usepackage{fancyhdr}
\usepackage{lastpage}
\usepackage{hyperref}
\usepackage{datetime2}
\usepackage{setspace}
\usepackage{xcolor}

\hypersetup{
  colorlinks=true,
  linkcolor=blue,
  urlcolor=blue,
  citecolor=blue,
  pdfauthor={第一組},
  pdftitle={阿根廷經濟改革對匯率的影響(整合專業報告)}
}

\graphicspath{{fig/}}

\pagestyle{fancy}
\fancyhf{}
\fancyhead[L]{阿根廷經濟改革對匯率的影響}
\fancyhead[R]{\leftmark}
\fancyfoot[C]{\thepage/\pageref{LastPage}}

% 表格欄位輔助
\newcolumntype{Y}{>{\raggedright\arraybackslash}p{0.22\textwidth}}
\newcolumntype{Z}{>{\centering\arraybackslash}p{0.14\textwidth}}

\title{\bfseries 阿根廷經濟改革對匯率的影響\\(整合專業報告)}
\author{國際金融理論與實務 第一組\\
113ZB1067 陳忠新 \quad 113ZB1073 陳文玲 \quad 113ZB1077 林昀臻 \\
113ZB1078 蕭舜文 \quad 113ZB1096 陳昱熙 \quad 113ZB1094 林雅情 \\
113ZB1109 王嘉興 \quad 113ZB1108 郭正昕 \quad 113ZB1117 陳玟玲 \quad 113ZB1111 高志鋼}
\date{版本:auto \quad 日期:\DTMdate{\the\year-\the\month-\the\day}}

\begin{document}
\maketitle
\tableofcontents
\thispagestyle{fancy}
\onehalfspacing

\section*{摘要}
\addcontentsline{toc}{section}{摘要}
本報告整合阿根廷自 2015 年以來三任政府之匯率、財政與貨幣政策,評估其對通膨、外匯儲備、資本流動與社會分配的傳導效果。主要發現:一、短期性多重匯率可緩解外匯壓力,卻因利差擴大與套利扭曲而削弱政策傳導;二、以央行融資赤字導致的高通膨與貶值互相強化,信用赤字使通膨預期黏著;三、一次性貶值與緊縮在短期改善外儲、壓抑通膨動能,但社會成本高、政治承受度有限。建議中期採「漸進匯率整合+財政紀律法制化+定向社會補償」,並在條件允許時審慎評估部分美元化作為信任錨,以建立「穩定—投資—生產率」正向循環。

\section{引言與研究設計}
\subsection{研究背景與動機}
阿根廷長期在高通膨、貨幣替代與低外儲緩衝間擺盪,政策於固定與浮動、單一與多重匯率之間反覆切換,導致預期不穩與信用受損。本研究以 2015 年以降之政策循環為窗口,連結理論與實證,剖析政策組合如何透過匯率機制影響通膨與外儲,並回到社會分配之後果。

\subsection{研究問題與貢獻}
\begin{itemize}[leftmargin=1.5em]
  \item 不同政府之匯率與財政—貨幣組合如何影響通膨、外儲與產業配置?
  \item 多重匯率與資本管制的邊際效益與隱性成本為何?
  \item 美元化(部分/完全)的可行性條件與風險如何評估?
\end{itemize}
本文貢獻在於將政策歷程、理論框架與可追溯資料整合,形成分情境的可操作建議。

\subsection{方法與資料}
採制度比較、事件分析與指標時序觀察。核心指標含官方與平行匯率、CPI 通膨、外匯儲備、一次性調整與序貶節奏、貿易與資本流、失業與貧困率。資料來源為 INDEC、BCRA、IMF 與權威媒體;圖表見第八節前之「圖表」區。

\section{文獻回顧與理論架構}
\subsection{三元不可能與匯率制度選擇}
在資本流動自由、貨幣政策獨立、匯率穩定三者不可兼得下,選擇組合決定了政策約束。阿根廷經驗顯示,缺乏跨期財政—貨幣承諾時,固定匯率難以持久。

\subsection{通膨預期與貨幣替代}
高通膨與政策不連續誘發美元計價與外幣資產配置,使本幣政策效果遞減,形成「低信任—高替代」的負向循環。

\subsection{多重匯率與資本管制}
多重匯率短期緩解外儲壓力,但放大利差與套利,扭曲資源配置並降低透明度。

\subsection{美元化理論取徑}
部分美元化可提升交易穩定與通膨治理可信度;完全美元化有助快速降通膨但喪失匯率緩衝與最後貸款人機能,對財政紀律與勞動市場彈性要求高。

\section{政策與制度演進(2015–至今)}
\subsection{馬克里政府(2015–2019)}
解除外匯管制、恢復外部融資與和解債權訴訟,短期修復市場信心;但財政約束不足,外債脆弱性上升,2018 年披索急貶並求助 IMF。

\subsection{費爾南德斯政府(2019–2023)}
防禦性干預下,多重匯率與價格管制擴張,疫情期間財政擴張加劇通膨;雖避免立即性外債危機,然貨幣穩定與投資環境惡化。

\subsection{米雷伊政府(2023–)}
一次性貶值與序貶、緊縮與鬆綁管制並進,初期外儲回升、通膨動能回落;社會代價高且政治阻力顯著,需配定向補償維持可持續性。

\section{匯率—通膨—外儲:傳導機制與觀察}
\subsection{指標與來源}
匯率(官方、平行):BCRA 與市場行情;通膨:INDEC CPI;儲備:BCRA、IMF。

\subsection{描述性時序與關聯}
一次性貶值後通膨短期加速,若搭配財政收斂與名目錨(如序貶規則),通膨可逐季回落;外儲在匯率調整與融資改善下回升。

\begin{figure}[h]
  \centering
  \includegraphics[width=\textwidth]{fig_fx.png}
  \caption{官方與平行匯率及匯差(資料:data/argentina\_macro.csv)}
  \label{fig:fx}
\end{figure}

\begin{figure}[h]
  \centering
  \includegraphics[width=\textwidth]{fig_infl_fx.png}
  \caption{通膨與官方匯率(資料:INDEC、匯率示例值)}
  \label{fig:infl}
\end{figure}

\begin{figure}[h]
  \centering
  \includegraphics[width=\textwidth]{fig_reserves.png}
  \caption{國際儲備走勢(資料占位,請以 BCRA 月底值更新)}
  \label{fig:res}
\end{figure}

\section{多重匯率與美元化之評估}
\subsection{多重匯率的動機與成本}
動機在於保護民生與優先部門、緩解外匯短缺;成本為稅基流失、套利擴大、出口激勵扭曲、制度透明度受損。

\subsection{美元化選項與條件}
條件包含外儲覆蓋、本幣資產切換、銀行體系穩健與最後貸款人安排。部分美元化提升可信度但仍受美元缺口限制;完全美元化快速降通膨但喪失緩衝。

\subsection{國際經驗對照}
厄瓜多與薩爾瓦多顯示美元化降低通膨與匯率風險溢價,但成長表現取決於治理與結構改革。

\section{社會與產業影響}
一次性貶值與緊縮短期壓抑實質工資,若缺乏定向補償與就業轉換支持,貧困率上升;中期效果取決於通膨回落與投資回升。產業面,出口導向與進口可替代部門受惠,依賴外幣債務與進口中間財之部門承壓。

\section{國際比較與可移植經驗}
制度可信度與治理透明度是穩定預期先決條件;匯率選擇需與財政—工資機制一體化設計;金融體系改革應與產業升級、貿易促進同步。

\section{政策情境與風險矩陣}
\subsection{基準情境(政策連貫)}
序貶規則透明、財政紀律與逐步縮表、清晰名目錨;結果為通膨回落、外儲回升、匯差收斂、增長溫和恢復。

\subsection{悲觀情境(紀律鬆動/外部衝擊)}
政策不連續、外部融資惡化或大宗商品下行;結果為匯率與通膨再度共振、外儲承壓、社會反彈升高。

\subsection{樂觀情境(改革深化/外需改善)}
法制化財政規則、國營改革推進、出口擴張與投資回流;結果為風險溢價下降、銀行體系恢復長端融資、就業改善。

\section{資料表與對照}
\subsection*{表一:2015–2025 匯率、通膨與外儲變化}
\addcontentsline{toc}{subsection}{表一:2015–2025 匯率、通膨與外儲變化}
\begin{longtable}{lccc}
\toprule
年份 & 平均匯率(披索/美元) & 通膨率(\%) & 外匯儲備(億美元)\\
\midrule
2015 & 9.0   & 26  & 300\\
2018 & 28.1  & 47  & 550\\
2020 & 70.1  & 36  & 420\\
2022 & 150.0 & 94  & 300\\
2023 & 365.0 & 142 & 330\\
2024 & 800.0 & 50  & 410\\
2025上半年 & 850.0 & 30  & 450\\
\bottomrule
\end{longtable}

\subsection*{表二:2015–2025 社會指標變化}
\addcontentsline{toc}{subsection}{表二:2015–2025 社會指標變化}
\begin{longtable}{lccc}
\toprule
年份 & 失業率(\%) & 貧困率(\%) & 實質工資年變化(\%)\\
\midrule
2015 & 7.0 & 29 & 2\\
2018 & 9.5 & 32 & -3\\
2020 & 10.8 & 38 & -7\\
2022 & 8.5 & 42 & -10\\
2023 & 9.2 & 41 & -5\\
2024 & 10.1 & 43 & -2\\
2025(預估) & 9.5 & 40 & 1\\
\bottomrule
\end{longtable}

\subsection*{美元化案例比較(阿根廷 vs 厄瓜多 vs 薩爾瓦多)}
\addcontentsline{toc}{subsection}{美元化案例比較}
\begin{longtable}{lcccc}
\toprule
國家 & 實施年份 & 外儲(億美元) & 實施後首年通膨(\%) & 主要經濟結果\\
\midrule
厄瓜多 & 2000 & 12 & 96 $\to$ 37 & 匯率穩定、物價下降、增長回復\\
薩爾瓦多 & 2001 & 20 & 4.3 $\to$ 3.0 & 低通膨但增長停滯\\
阿根廷(假設) & 2025 & 45 & 142 $\to$ 30(估) & 穩定與彈性取捨\\
\bottomrule
\end{longtable}

\section{結論與政策建議}
\textbf{結論。} 阿根廷症結在信任赤字與政策不連續。若無跨期的財政—貨幣承諾與透明治理,任何匯率制度難以長穩。\par
\textbf{建議。}
\begin{enumerate}[leftmargin=1.5em]
  \item 匯率:設定透明序貶規則與走廊,以外儲—通膨雙指標作為停損條款;逐季收斂多重匯率。
  \item 財政:法制化中期財政框架與赤字上限;重構補貼與稅支,擴大高乘數公共投資。
  \item 貨幣—金融:強化央行獨立性與資產負債表修復;重建本幣長端債市與市場化利率曲線。
  \item 社會:建立「社會調適基金」,落實定向現金轉移、技能再培訓與就業媒合。
  \item 美元化評估:以「外儲覆蓋、金融安全網、政治承受度」為門檻,審慎推進部分美元化。
\end{enumerate}

\section*{參考資料}
\addcontentsline{toc}{section}{參考資料}
官方資料:
\begin{itemize}[leftmargin=1.5em]
  \item INDEC(CPI 與社會指標):\url{https://www.indec.gob.ar/}
  \item BCRA(外匯、儲備與政策):\url{https://www.bcra.gob.ar/}
  \item 阿根廷政府開放資料:\url{https://datos.gob.ar/}
\end{itemize}
國際機構:
\begin{itemize}[leftmargin=1.5em]
  \item IMF 阿根廷國別頁:\url{https://www.imf.org/en/Countries/ARG}
\end{itemize}
媒體與行情:
\begin{itemize}[leftmargin=1.5em]
  \item Reuters 阿根廷專題:\url{https://www.reuters.com/world/americas/argentina/}
  \item Bloomberg 匯率與政策動態:\url{https://www.bloomberg.com/markets/currencies}
\end{itemize}

\end{document}